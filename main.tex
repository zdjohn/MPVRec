%%%%%%%%%%%%%%%%%%%%%%%%%%%%%%%%%%%%%%%%%%%%%%%%%%%%%%%%%%%%%%%%%%%%%%%%%%%%%%%%%%
%% For technical support please email: ykoh@wspc.com.sg (or) rajesh@wspc.com.sg %%
%% The content, structure, format and layout of this style file is the          %%
%% property of World Scientific Publishing Co. Pte. Ltd.                        %%
%% Copyright 2014 by World Scientific Publishing Co.                            %%
%% All rights are reserved.                                                     %%
%%                                                                              %%
%% Proceedings Trim Size: 9in x 6in                                             %%
%% Text Area: 7.35in (include runningheads) x 4.5in                             %%
%% Main Text is 10/13pt                                                         %%
%% Last Modified: 24-01-2014                                                    %%
%%%%%%%%%%%%%%%%%%%%%%%%%%%%%%%%%%%%%%%%%%%%%%%%%%%%%%%%%%%%%%%%%%%%%%%%%%%%%%%%%%
%
%\documentclass[wsdraft]{ws-procs9x6}  % to draw border line around text area
%\documentclass[wssquare]{ws-procs9x6} % for citations in square brackets (consult your editor before picking up this style)
\documentclass{ws-procs9x6}            % default, citations in superscript
\usepackage{lmodern}

\begin{document}

\title{Recommender systems with heterogeneous information network for cold-start items}

\author{Di Zhang, Qian Zhang, Guanquan Zhang, Jie Lu}

\address{\textit{Decision Systems and e-Service Intelligence Laboratory,} \\
\textit{Centre for Artificial Intelligence}\\
University of Technology Sydney, Australia \\
Di.Zhang-7@student.uts.edu.au, Qian.Zhang-1@uts.edu.au \\
Guangquan.Zhang@uts.edu.au, Jie.Lu@uts.edu.au}

\begin{abstract}
Recommender System is a widely adopted research topic in real-world applications. Collaborative Filtering (CF) and Matrix based approach has been the main stream methodology for the past decade in both implicit and explicit recommendation tasks. One challenge most of above methods having is "Cold Start" problem, that is when newly emerged data set has little or no available interactions for training. Some model based CF condense user/item data into latent space, through learning process such as Factorization Machine. However, in case of constant cold start problem, those approach can be ineffective and costly. In this paper, we propose MERec, a novel approach that adopts graph meta-path embedding to learns item/user features independently besides learning user-item interactions. It allows unseen data points to be incorporated as part of user/item learning process, hence effectively reduce the impact in cold start problem for new or sparse dataset.
\end{abstract}

\keywords{recommender system; cold-start; heterogeneous information network.}

\bodymatter

%!TEX root = ./main.tex
\section{Introduction}

Recommender system is an indispensable technology in this big data era \cite{lu2015recommender}. It helps us to find personalized products or services from the ever-increasing information and make our life more efficient and focused. Collaborative Filtering (CF) and matrix factorization methods are the dominant recommendation approach for the past decades. These methods predict user's preference base off user-item interactions. Consequently, a warming up period is commonly required for CF based approach to over come the data sparsity problems, by accumulating interactions records.

In general, matrix factorization approach projects user/item features into a latent space, then calculate user-item similarity based on their condensed latent vector values. Such methods are not suited in dealing brand new items, which have a wide range of use case in real-world scenarios. A known workaround is running a routine end to end re-train process to pick up new interactions, which can be inefficient and costly. Content-based or hybrid recommendation approaches are one way to overcome the problem. It is proved to be challenging when dealing large categorical features, such as exponential computation complexity increase, as well as introducing noise.

Recently, many embedding methods are developed via Heterogeneous Information Network (HIN) based approaches \cite{mao2016multirelational,wang2016member}. Graph data structure enables continuously evolving information. The embedding process learns item/users' representations based on nodes' local and global structure, via propagation process such as random walk, such as Node2vec \cite{grover2016node2vec} and DeepWalk \cite{perozzi2014deepwalk}. However, random walks based presentation learning are known to be more biased to highly connected nodes \cite{sun2011pathsim}. This problem is alleviated by meta-path based approaches \cite{dong2017metapath2vec} especially for recommendation with explicit feedback. However, not many research emphasize on the data sparsity problem and they cannot deal with brand new items. Moreover, many real-world applications are focusing on implicit feedback, such as optimizing click through rate. This remains to be a challenging problem to be solved.

In this paper we propose a Meta-path Embedding based Recommendation (MERec), a novel approach that utilizing meta-path based presentation learning on HIN. This approach is primarily focusing on solving the cold start problems when data is continuously evolving. It is also designed to be adaptive to newly emerged data, so that recommendation could be made even there is zero user-items interaction available. By translating domain experts relevancy rules into meta-path set, our approach allow a controlled random walk to reduce computation intensity and improve on end result. 

We Identify the paper contributions as following:
\begin{itemize}
    \item A novel meta-path based embedding method to effectively handle cold-start problem in  recommendation with implicit feedback;
    \item Our approach uses meta-path based item feature embedding, then learns user presentation in subsequent step using interactions data, allowing model to predict recommendations on brand new items. 
    \item Experiments are conducted on 3 different data conditions: pure cold start, sparse data, non-sparse data. the end result shows consistent performs across all 3 scenarios;
\end{itemize}

The rest of this paper is constructed as follows. In Section 2, we discuss the related research and relevant definition that helped our research. In Section 3, we explain our framework and related algorithms. In Section 4, we show our experiment design and result compared with other baselines in cold-start scenario and sufficient data scenario. Lastly, we give conclusion and future directions in Section 5.

%!TEX root = ./main.tex
% \section{Related Works}
% Graph based presentation learning is a actively researched topic in areas such as node classification, clustering, and similarity search. As intuitively, objects similarity can be measured based on node distance and density of given graph/sub-graph. Studies such as P-PageRank \cite{bahmani2010fast}, SimRank \cite{jeh2002simrank}, leverage homogeneous network structures. While research like PathSim \cite{sun2011pathsim}, HeteSim \cite{shi2014hetesim} manages to taking different types of objects and links in to data mining process, so that different semantic meaning across different types of nodes can be learned respectively. With the advancement of NLP research, i.e. Word2Vec \cite{mikolov2013efficient}. Embedding methods are introduced into node presentation learning. For example, node2vec \cite{grover2016node2vec} adopts CBOW and SkipGram into random walk process. Subsequently, Metapath2Vec \cite{dong2017metapath2vec} introduced a guided random walk approach to help reduce the bias of random walk, by introducing a set of predefined meta-path. 

% Since `The Netflix Prize' \cite{bennett2007netflix}. Memory based CF measures similarity by using Pearson Correlation or vector cosine similarities to derive predictions though a aggregated nearest neighbor mechanism. Model based CF leverage data mining techniques by learning user/item latent features. As knowledge graph and especially Graph Neural Networks gaining traction \cite{wu2019comprehensive}, there are a number of HIN based techniques being developed such as item2vec \cite{Barkan2016}, entity2vec \cite{palumbo2017entity2rec}. and HErec \cite{shi2018heterogeneous} towards user-item ratings predictions. Though, We are not able to see much research results on the implicit recommendation front. 

% Hence, we propose MERec, a framework that is motivated on solving cold start problems on implicit predictions using HIN. 

\section{Preliminaries}\label{3PD}
Following definitions are preliminaries of our approach.

\begin{definition}[Heterogeneous Information Graph]
An information graph is $G = (V,E)$, where $V$ is the set of nodes (or entities) of the graph. $E$ is the set of edges connecting the nodes in $V$, $E \subset V \times V$.
Entity and link type mapping $\phi$: $V \rightarrow A$, $\varphi$: $E \rightarrow R$, where $A$ and $R$ denote the sets of predefined entity and link types, and $|A| > 1$ or $|R| > 1$ indicating that there are more than one type of nodes and inter nodes relationships.
\end{definition}

\begin{definition}[Meta-Path]\label{def:metaPath}
A Meta-Path $\mathcal{P}$ is a path defined on the graph schema $T_G = (A, R)$. \newline
Meta-Path $\mathcal{P}$ is denoted as $A_1 \xrightarrow{\text{r1}} A_2 \xrightarrow{\text{r2}} \dots \xrightarrow{\text{rn}} A_n$. 

Relationship $R$ is denoted as $r1 \bullet r2 \bullet ... rn$ for different types of relationship between different types of entity nodes, where $\bullet $ denotes composition operator or relations.
\end{definition}

Meta-Path $\mathcal{P}_1: A_1 \xrightarrow{r_1} A_2$ is the Meta-Path connects source entity type $A_1$ and target $A_2$.
Similarly, $\mathcal{P}_2: A_2 \xrightarrow{r_2} A_3$, $\mathcal{P}_2$ is the Meta-Path between source entity type $A_\text{2}$ and target entity type $A_\text{3}$.
Here we call $\mathcal{P}_1$ and $\mathcal{P}_2$ are contactable. Then $\mathcal{P}_1$ and $\mathcal{P}_2$ can be combined as $\mathcal{P}_{1,2}: A_1 \xrightarrow{r_1} A_2 \xrightarrow{r_2} A_3$. For example, $Movie \rightarrow Director$ and $Director \rightarrow Movie$ can be combined to $Movie \rightarrow Director \leftarrow Movie$. 


\section{Methodology}
Most of the time, different business have unique data characteristics. To build a recommender system, especially, require customized algorithm for model tuning.  Having a sound business domain knowledge is very helpful in better using its data, creating the performant prediction model. In this paper, we propose a framework MVRec (meta-path embedding based Recommendation) that allow recommendation models to learn item features embedding by adapting custom domain knowledge in a generic way. 

\begin{figure*}[!t]
    \centering
    \includegraphics[width=0.8\textwidth]{figs/fig1.png}
    \caption{heterogeneous information graph based on item features}\label{fig:fe-graph}
\end{figure*}

One of the key idea of our approaches is inspired by recent research on graph-based embedding (reference) Instead of using random walk to learn low-dimensional item feature presentation, our approach introducing a pre-defined set of meta-path to create user interfered bias to control the walk random process from one node to another in HIN. The benefit of such practice can be review from 2 different fronts: 

\begin{enumerate}
    \item Intuitively, introducing user-defined sets of meta-path, allows the model to accommodate user background knowledge into the random walk process. This can prevent over-fitting on some common nodes that overcrowded by connections. i.e. comedy is the top genre in movie-lens data sets. Over 20 \% of the movie are comedies. However, most of the time user would not choose a movie purely based on genre alone.
    \item Categorical data is common but tricky to adapt into model learning. A common practice of handling categorical data is using one-hot (reference) encoding. However, it can quickly become a problem, when the categorical data has a lot of unique values. For example, the movie-lens dataset X actors, Y directors, X genre and so on. Using one-hot encoding normally leads to super large matrix size and high data sparsity.
\end{enumerate}

Consequently, as the item embedding is learned based on item features rather than depending on user-item interactions. This allows model to handle cold start problem and data sparsity problem effectively. 

\subsection{Problems and Definitions}\label{3PD}
Following definitions are used for describe our approach MVRec.

\begin{definition}[Heterogeneous Information Graph]
A information graph is $G = (V,E)$, where $V$ is the set of nodes (or entities) of the graph. $E$ is the set of edges connecting the nodes in $V$, $E \subset V \times V$. \newline
Two mapping functions: Entity type mapping function $\phi$: $V \rightarrow A$, and link type mapping function $\varphi$: $E \rightarrow R$, where $A$ and $R$ denote the sets of predefined entity and link types, and $|A| > 1$ or $|R| > 1$ indicating that there are more than one type.
\end{definition}

For example, we use movie attribute information to enrich user-item ratings in Movielens data set using graph model as shown in Fig. %\ref{fig:enrich}
In this paper we use \textit{``actors'', ``director'', ``writer'', ``genre''} etc. as different types of nodes. These nodes are in the same graph of user nodes and item nodes. How to process with different nodes in one graph? Graph schema indicates how different types of entities link with each other. It serves as a template to describe the structure as well as the semantic relationship between object types.

Between two entities $x$ and $y$, there are different paths connecting the two nodes. As for the case of Movielens, \textit{MovieA} and \textit{MovieB} can be connected via \textit{MovieA-Actor-MovieB} or \textit{MovieA-Director-MovieB} or \textit{MovieA-User-MovieB} path connections. We call those path, which containing multiple entities, Meta-Path.

\begin{definition}[Meta-Path]\label{def:metaPath}
A Meta-Path $\mathcal{P}$ is a path defined on the  graph schema $T_G = (A, R)$. \newline
Meta-Path $\mathcal{P}$ is denoted as $A_1 \xrightarrow{\text{r1}} A_2 \xrightarrow{\text{r2}} ... \xrightarrow{\text{rn}} A_n$. 

Relationship $R$ is denoted as $r1 \bullet r2 \bullet ... rn$ for different types of relationship between different types of entity nodes, where $\bullet $ denotes composition operator or relations.
\end{definition}

For Meta-Path $\mathcal{P}_i$ shares same graph schema, there could also be multiple path $p$ connecting source entity $a_i$ to target ${a}_{i+1}$. Each path $p$ inside Meta-Path $\mathcal{P}_i$ is a path instance, $p \in \mathcal{P}_i$. The number of path instances $p$ between $a_i$ and $\bm{a}_\text{i+1}$, is called path count. Reverse Meta-Path $\mathcal{P}^{'}$ is the reversed relation sequence of $\mathcal{P}_i$, if $\mathcal{P}^{'}$ is the reverse path of $\mathcal{P}$ in $T_G = (A, R)$, reverse path is denoted as $\mathcal{P}^\text{-1}$. \newline

Meta-Path $\mathcal{P}_1: A_1 \xrightarrow{r_1} A_2$ is the Meta-Path connects source entity type $A_1$ and target $A_2$.
Similarly, $\mathcal{P}_2: A_2 \xrightarrow{r_2} A_3$, $\mathcal{P}_2$ is the Meta-Path between source entity type $A_\text{2}$ and target entity type $A_\text{3}$.
Here we call $\mathcal{P}_1$ and $\mathcal{P}_2$ are contactable. Then $\mathcal{P}_1$ and $\mathcal{P}_2$ can be combined as $\mathcal{P}_{1,2}: A_1 \xrightarrow{r_1} A_2 \xrightarrow{r_2} A_3$. For example, $Movie \rightarrow Director$ and $Director \rightarrow Movie$ can be combined to $Movie \rightarrow Director \leftarrow Movie$. 

\subsection{Meta-path Based Random Walk}\label{3MF}

For items such as movies, books, music, there are a number of factors impacting users' decision. 

Instead of taking the one-hot encoding approach, treating each category value as a feature column. We propose to treat each feature category as an independent node type. Putting different feature type nodes and items together, here we formed a heterogeneous information graph. 

Next, we define sets of meta-path that is known to be effective factors for item-item similarities, based on the expert domain knowledge or based on feature analysis, such as, PCA for computation reduction purposes. For example, movie choice is closely linked to directors and its casts. Thus $Movie \rightarrow Director \leftarrow Movie$, $Movie \rightarrow Actor/Actress \leftarrow Movie$  can be very important meta-path in deciding how similar 2 movies are. We use those insights as a guideline to from a heterogeneous information graph based on item features. As shown in Fig. \ref{fig:fe-graph}

Each meta-path can derive a item-item similarity matrix $\mathbb{R}_i(\mathcal{P}_i)$, each different meta-path can be regarded as bias toward different feature aspects, so items co-occurrence can be learned separately under different meta-path.
as a result, item-item similarity score can be calculated by normalized meta-path similarity times meta-path weight.

\begin{equation}\label{itemsim}
    \mathcal{S}(v_i,v_j) = 
    \begin{cases}
         \sum\limits_{\substack{n=1}}^{n} \mathcal{R}_ij(\mathcal{P}_n,{W_n}),& \text{if } (v_{i}, .., v_{j}) \in \mathcal{P} \\
         0,              & \text{otherwise}
     \end{cases}
\end{equation}

$\mathcal{S}(v_i,v_j)$ stands for similarity between items $v_i$ and $v_j$ which shares the same node type. $R_ij()$ is a similarity function, where $\mathcal{P}_n, {W_n}$ stands for individual meta-path and its weights respectively. This end result provides guidance for the random walkers on our heterogeneous graph of item features. $(v_{i}, .., v_{j})$ is denoted as a meta-path instance, where $v_i$ is the starting node and $v_j$ the end node.

Given a heterogeneous graph $G = (V,E)$, and a meta-path set $[\mathcal{P}_1, \mathcal{P}_2, ... \mathcal{P}_n]$, the probability of transition is defined as following:

\begin{equation}\label{hetewalker}
    P(v_{i+1},\mathcal{P},w)= 
        \begin{cases}
            p({N^{t+1}(v_{i}^t)}),& \text{if } (v_{i+1}, .., v_{i}^t) \in \mathcal{P} \\
            0,              & \text{otherwise}
        \end{cases}
\end{equation}

$t$ is denoted as $t^th$ steps, as the walker traversing through the graph.
$p({N^{t+1}(v_{i}^t)})$ is a $softmax$ function on top of the neighbors of node $v_{i}^t$. 
that is:

\begin{equation}\label{softmaxwalker}
    p({N^{t+1}(v_{i}^t)}) = \frac{Exp(\mathcal{S}(v_i,v_j))}{\sum\limits_{\substack{n=1}}^{n} {Exp(\mathcal{S}(v_i,v_j)})}
\end{equation}

we enable skip-gram to learn the presentation of given node $v$:

\begin{equation}\label{skipgram}
    arg max
    \sum\limits_{\substack{v \in V}}
    \sum\limits_{\substack{c \in N(v)}}
    log p({c|v;\theta})
\end{equation}

$log p({c|v;\theta}))$ is the softmax function as defined in \cite{mikolov2013distributed} \cite{mikolov2013efficient}. In our approach, we substitute $log p({c|v;\theta}))$ with softmax function defined in equation \ref{softmaxwalker}. $c$ is denoted as $context$, in graph structure setting, $c$ is the neighboring nodes of given node $v$, i.e. $N(v)$. 

Finally, we introduce 3 hyper parameters to learn the item vector representation. $d$ for dimension size, $x$ for number of walks, and $l$ for depth of each random walk. 


\subsection{Pure Cold Start: MVRec-PCC}\label{3PCC}
One of common real world recommendation scenario is recommending new item to potential audiences/customers. For example, product launch, movie premier. Listings portal, such as, job listings (e.g. LinkedIn) and real estate listings (Zillow) faces constant pure cold start problem, where historical user-item interaction is not available to learn. 

In order to solve pure cold start problem, commonly feature based (content based) are one of limited way to make recommendation. 
Additionally, we adopt Pearson Correlation Coefficient (PCC) to be in combination with MVRec through the recommendation process. 
PCC is a popular measurement approach as defined in equation \ref{pcc}, unlike Cosine Similarity, PCC takes the difference of mean and variance between users' rating scale into account.

\begin{strip}
    \begin{equation}\label{pcc}
        S_{u_i,u_j}^{PCC} = 
        \dfrac{
            \sum_{v \in V_{u_i} \cap V_{u_j}} (r_{u_i,v}-\overline{r_{u_i}}) 
            * 
            (r_{u_j,v} - \overline{r_{u_j}})
        }
        {
            \sqrt{\sum_{v \in V_{u_i} \cap V_{u_j}} (r_{u_i,v}-\overline{r_{u_i}})^2} 
            * 
            \sqrt{\sum_{v \in V_{u_i} \cap V_{u_j}} (r_{u_j,v}-\overline{r_{u_j}})^2}
        }
    \end{equation}
\end{strip}

accordingly, we can normalized the user-item rating based on normalized user rating $\overline{u}$ overall:
\begin{equation}\label{pcc}
    NS(r_{u_i,v_j}) =  S_{u_i,\overline{u}}^{PCC}
\end{equation}

Consequently, we get the top ranked user by calculating as following:

\begin{algorithm}
    \caption{ranked users list}
    \begin{algorithmic}[1] 
    \REQUIRE~~\\ %Input
    New Item: $v$\\
    Existing user item ratings: ${R_{mxn}}$\\
    Items vectors: $\mathcal{V}$\\
    Top X nearest items: $X$\\
    Top K ranked users list rank: $K$\\
    \ENSURE~~\\ %Output
    \STATE Compute nearest items $TopN(v,\mathcal{V})$ based on $\mathcal{V}$ and $v$
    \STATE \textbf{for} x=1 : X
    \STATE \quad Compute normalized rating per user of $v_x$
    \STATE \quad Ranked users $RankedList(v_x) \leftarrow (u_i, NS(r_{u_i,v_x}))$
    \STATE \quad $TopKList(v) \leftarrow$ Top $K$ (user,rating) in $RankedList(v_x)$
    \STATE $RankedList(v)$ = \textbf{from} $TopKList(v)$ \textbf{group by} user \textbf{select} (user,$\sum{rating}$) \textbf{order by} $\sum{rating}$
    \RETURN Top K of $RankedList(v)$
    \end{algorithmic}\label{alg:1}
\end{algorithm}

In Section \ref{4_experiment}, we would show more detailed comparison results with traditional categorical one-hot encoding approach

\subsection{Sparse Recommendation: MVRec-BPR}\label{3BPR}
One of the other common cold start problem, that we commonly facing is sparse data. Matrix Factorization or CF based approach, try to learn user and item presentation (i.e. $U, V$) in a latent space jointly, normally suffers from data sparsity problem.


Here we propose, instead of learning $U, V$ jointly, we replace $V$ with meta-path based vectors $Vec(v)$. This approach provides several benefits:
\begin{enumerate}
        \item Reducing learning complexities
        \item Taking both item features and user-item interactions into account
        \item $Vec(v)$ is less impacted when user-item interactions are sparse 
\end{enumerate}

We use BPR as our optimization objective:

\begin{equation}\label{skipgram}
    arg max (u_i \cdot Vec(v)-u_j \cdot Vec(v)) - \dfrac{\lambda}{2}tr(U^TU)
\end{equation}


Based on the experiment result in Section \ref{4_experiment}, we see MVRec out performs widely used CF+ BPR by a large margin in sparse dataset. It also gained equivalent or minor advantage when the dataset is less sparse in non-cold start settings.




\section{EXPERIMENTS AND RESULTS}\label{4_experiment}
We have run a number of experiments based on several common real world scenarios, and we compared our approach with other popular methods through the experimentation:
\begin{itemize}
    \item How effective MERec performs in $Pure$ $Cold$ recommendation tasks. i.e. new product release, target audience prediction.
    \item How effective When MERec handle $Sparse$ dataset, in common recommendation tasks
    \item How dose MERec compare with other state-of-the-art methods
    \item illustrate the adaptiveness of MERec 
\end{itemize}

\subsection{Data Set}
Our experiments is based on HetRec 2011 \cite{CantadorRecSys2011} data set, which is a extension of MovieLens10M data enriched with content data collected from IMDb and Rotten Tomatoes.
First, we prune the data. For movies, we only picks the movie with at least 10 distinctive user views. we also limit top 3 actors/actresses to be associated. In terms of Tags, we are only use tags information which being share by more than 1 movies. In the end, we concluded 6 different types of $node$ which can be used in heterogeneous information graph, shown as following:
% # users: 2113, # movies: 6829, # medium rating: 3.5, # ratings: 841910, #tags: 2356, # actors: 9295, # directors: 2762 # genres: 19
\begin{center}
    \begin{tabular}{|l|l|l|l|l|}
    \hline
     \textbf{Node Type} & \textbf{Unique Number} \\ \hline
     Users &  2113\\ \hline
     Movies &   6829 \\ \hline
     Tags &  2356 \\ \hline
     Actors &  22256 \\ \hline
     Directors &  2762 \\ \hline
     Genres &  19 \\ \hline
    \end{tabular}
\end{center}

Additionally, following 5 relationships is defined as edges, which are $Movie-Users$, $Movie-Tags$, $Movie-Actors$, $Movie-Directors$, and $Movie-Genres$:

\begin{center}
    \begin{tabular}{|l|l|l|l|l|}
    \hline
     \textbf{Edge Type} & \textbf{Total Number} \\ \hline
     $Movie-Users$ &  841910\\ \hline
     $Movie-Tags$ &   9362 \\ \hline
     $Movie-Actors$ &  2356 \\ \hline
     $Movie-Directors$ &  6893 \\ \hline
     $Movie-Genres$ &  15119 \\ \hline
    \end{tabular}
\end{center}

For the simplicity of the experiment, we weigh all of our meta-path equally, as it is sufficient in illustrating the effectiveness of MERec. On the other hand, we do acknowledge that, those hyper-parameters can be further tuned to improve the result to tailor different problem domains. 


\subsection{Pure cold start: MERec + Pearson Correlation Coefficient (PCC)}
When it comes to $Pure$ $Cold$ start problem, it means, new items are being recommended to users which had never seen before. as a result are are splitting our data set based on time.
we use movies prior 2008 as training data set (6724 movies), and 2008~2011 as testing data set (193 movies). Here we compare MERec with Content-Based (C.B.) recommendation, since CF approach will be very limited when there is no user-item interactions available.

The choice of meta-path in this experiment are: $Movie-Tags-Movie$, $Movie-Actors-Movie$, $Movie-Directors-Movie$, and $Movie-Genres-Movie$. We only choose feature related meta-path. By doing this way, we can not only can demonstrate MERec produces superior Precision and Recall. It also shows meta-path based movie content representation are more effective when dealing with large amount of categorical data. 
We also ran a separate experiment to add $Movie-Users-Movie$ meta-path as part of embedding process. we can see the result comparison as below:


    \textbf{pure cold start recommendation}
    \begin{center}
    \begin{tabular}{|c | c | c | c | c | c | c|} \hline

    \textbf{Model} & \textbf{Precision\@1} & \textbf{Recall@1} & \textbf{Precision@5} & \textbf{Recall@5} & \textbf{Precision@10} & \textbf{Recall@10} \\ \hline
    \text{C.B.} &  0.1390 & 0.0027 & 0.1229 & 0.0146 & 0.1481 & 0.0334 \\ \hline
    \text{MERec+PCC} & \textbf{0.3743} & 0.0087 & 0.2770 & 0.0350 & 0.2465 & 0.0604\\ \hline
    \text{MERec+PCC (incl. interactions)} & 0.34759 &  \textbf{0.0096} &  \textbf{0.3037} &  \textbf{0.0402} &  \textbf{0.2711} &  \textbf{0.0665} \\ \hline
    \end{tabular}
\end{center}

Here we can observe a further improved to the end result, after including past user-item interaction into the embedding process.


\subsection{Cold start with Sparse data: MERec + BPR}
In this section we are going to split data into 2 different density distribution to evaluate the effectiveness of MERec performance in both sparse and dense dataset. we also compares it with CF+BPR, which is being one of the most popular and effective algorithms.

First we split our dataset into 2 different random split portions: In sparse data scenarios, the data density set to be \text{1.1\%}, while in non-sparse experiment settings, the density is set at \text{2.3\%}

the comparison result are shown as flowing, In sparse data cases, we can see in average near 10\% performance boost in precision. While in non-sparse cases, we can still see equivalent or minor enhancement across precision and recall:

    \textbf{sparse data recommendation}
    \begin{center}
        \begin{tabular}{|l|l|l|l|l|l|l|}
        \hline
    
    \textbf{Model} & \textbf{Precision\@1} & \textbf{Recall@1} & \textbf{Precision@5} & \textbf{Recall@5} & \textbf{Precision@10} & \textbf{Recall@10} \\ \hline
         \text{CF+BPR (sparse)} &  0.6146 & \textbf{0.0039} & 0.5498 & 0.0126 & 0.5548 & 0.0265 \\ \hline
         \text{MERec+BPR (sparse)} & \textbf{0.6616} & 0.0036 & \textbf{0.6194} & \textbf{0.0163} & \textbf{0.6035} & \textbf{0.0308}\\ \hline
        \end{tabular}
    \end{center}
    



    \textbf{dense data recommendation}
    \begin{center}
        \begin{tabular}{|l|l|l|l|l|l|l|}
        \hline
    
    \textbf{Model} & \textbf{Precision\@1} & \textbf{Recall@1} & \textbf{Precision@5} & \textbf{Recall@5} & \textbf{Precision@10} & \textbf{Recall@10} \\ \hline
         \text{CF+BPR (dense)} & 0.4936 & 0.0027 & 0.5566 & 0.0175 & 0.5419 & 0.0336 \\ \hline
         \text{MERec+BPR (dense)} & \textbf{0.6720} & \textbf{0.0046} & \textbf{0.5858} & \textbf{0.0195} & \textbf{0.5575} & \textbf{0.0355} \\ \hline
        \end{tabular}
    \end{center}











%!TEX root = ./main.tex
\section{Conclusion and Future Work}
In this paper we propose MERec, a novel approach that can effectively predict newly emerged item/user when there is little or no interaction data available. It has a wide application in real-world recommendation tasks, such as job, real-estate, where cold-start is a common problem. 
While MERec allows user to define meta-path settings in terms of controlling random walk process, it also opens up possible enhancement on learning different weights automatically across multiple meta-path with in the HIN. Making the embedding inductive instead of transitive is another interesting topic worthy exploring.



% Unnumbered appendix sections can be obtained using \verb|\section*|.
% \section*{Acknowledgment}

% The preferred spelling of the word ``acknowledgment'' in America is without 
% an ``e'' after the ``g''. Avoid the stilted expression ``one of us (R. B. 
% G.) thanks $\ldots$''. Instead, try ``R. B. G. thanks$\ldots$''. Put sponsor 
% acknowledgments in the unnumbered footnote on the first page.

\bibliographystyle{ws-procs9x6} % for numbered citation & references
\bibliography{ref}

\end{document}

\end{document}