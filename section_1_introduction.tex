%!TEX root = ./main.tex
\section{Introduction}

Recommender system is an indispensable technology in this big data era. It helps us to find personalized products or services from the ever-increasing information and make our life more efficient and focused \cite{lu2015recommender}. Collaborative Filtering (CF) and matrix factorization are the dominant recommendation approach for the past decades. Its trained models predict user's preference base off user-item interactions. Consequently, a warming up period is commonly required for CF based approach to over come the data sparsity problems, by accumulating interactions records.

In general, matrix factorization approach projects user/item features into a latent space, then calculate user-item similarity based on their condensed latent vector values. They are not suited in dealing brand new items, though new product recommendation has very high demand in the real-world applications. A known workaround is running a routine end to end re-train process to pick up new interactions, which can be inefficient and costly. Content-based or hybrid recommendation approaches are one way to overcome the problem. It is proved to be challenging when dealing large categorical features, such as exponential computation complexity increase, as well as introducing noise.

Recently, many embedding techniques are developed via Heterogeneous Information Network (HIN) based approaches \cite{mao2016multirelational,wang2016member}. Graph data structure enables continuously evolving information. HIN based embedding process, such as Node2vec \cite{grover2016node2vec} and DeepWalk \cite{perozzi2014deepwalk}, learns item/users' representations based on nodes' local and global structure, via propagation process such as random walk. On the other hand, random walks based presentation learning are known to be more biased to highly connected nodes \cite{sun2011pathsim}. This problem is alleviated by meta-path based approaches \cite{dong2017metapath2vec} especially for recommendation with explicit feedback. However, not many research emphasized on the data sparsity and cold start problem, especially in dealing brand new items. Moreover, many real-world applications are focusing on implicit feedback, such as optimizing click through rate. This remains to be a challenging problem to be solved.

In this paper we propose a Meta-path Embedding based Recommendation (MERec), a novel approach that utilizing meta-path based presentation learning on HIN, and with a primarily focus on solving the cold start recommendation problems. It is also designed to be adaptive to newly emerged data, so that recommendation could be made even there is no user-items interaction available. By translating domain experts relevancy rules into meta-path set, our approach allow a controlled random walk to reduce computation intensity and improve on end embedding result. 

We Identify the paper contributions as following:
\begin{itemize}
    \item A novel meta-path based embedding approach to effectively handle cold-start problem in  recommendation with implicit feedback;
    \item Our approach uses meta-path based item feature embedding, then learns user presentation in subsequent step using interactions data, allowing model to predict recommendations on brand new items. 
    \item Experiments are conducted on 3 different data conditions: pure cold start, sparse data, non-sparse data. the end result shows consistent performs across all 3 scenarios;
\end{itemize}

The rest of this paper is constructed as follows. In Section 2, we discuss the related research and relevant definition that helped our research. In Section 3, we explain our framework and related algorithms. In Section 4, we show our experiment design and result compared with other baselines in cold-start scenario and sufficient data scenario. Lastly, we give conclusion and future directions in Section 5.