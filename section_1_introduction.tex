%!TEX root = ./main.tex
\section{Introduction}

Recommender system is an indispensable technology in this big data era \cite{lu2015recommender}. It help us to find personalized products or services from the ever-increasing information overloads and make our life more efficient and focused. Collaborative Filtering (CF) and matrix based approach are a dominant recommendation approach For the past decades. Most of the works require user-item interactions to be presented for training. Consequently, a warming up period is commonly required for CF based approach to over come the data sparsity problems, due to continuously evolving data.

Dimensional reduction method, by projecting user/item features into a latent space, is commonly used in model based CF approach. However, such methods are not suited in dealing brand new items, which have a profound use case in real-world scenarios. A known workaround is running a routine end to end re-train process to pick up new interactions, which can be inefficient and costly. Content or Hybrid approach, are one way to over come the problem. It is proven to be challenging when dealing large categorical features, such as exponential computation complexity increase, as well as introducing noise.

Recently, many embedding method is developed via HIN based approach \cite{mao2016multirelational,wang2016member}. Graph data structure enables continuously adding user/item feature related information. The embedding process item/users' representation can be learned based on nodes' local and global structure, via process such as random walk. For example, Node2vec, DeepWalk \cite{grover2016node2vec} \cite{perozzi2014deepwalk}. However, Random walks based presentation learning are known to be more biased to highly connected nodes. \cite{sun2011pathsim}. Meta-Path based approach \cite{dong2017metapath2vec} helps in controlling random walk probability for improved embedding result. Though, there are a number of research shown promising results in explicit recommendations problem. The cold start problem is little discussed, especially with the implicit recommendation domain. Problems, such as, optimizing click through rate without ratings have much wider real-world applications, and severer cold problem. 

In this paper we propose a Meta-path Embedding based Recommendation (MERec), a novel approach that utilizing meta-path based presentation learning on HIN. This approach is primarily focusing on solving the cold start problems when data is continuously evolving. The model is also designed to be adaptive to newly emerged data, so that recommendation could be made even there is zero user-items interaction available. By translating domain experts relevancy rules into meta-path set, our approach allow a controlled random walk to reduce computation intensity and improve on end result. 

We Identify the paper contributions as following:
\begin{itemize}
    \item a novel fusion approach to effectively handle fresh data problem in implicit recommendation problem;
    \item a controlled approach that capable in incorporating expert knowledge into training process;
    \item a framework that can effectively reduce end to end retraining interval when data keep evolving overtime;
\end{itemize}

The rest of this paper is constructed as follows: In Section II, we discuss the related research and relevant definition that helped our research. In Section III, we explain our framework and related algorithms. In Section IV, we show our experiment design and result comparison with other models in different cold and non-cold start scenario. Lastly In Section V, we discuss our finding and and explore future improve directions.