%!TEX root = ./main.tex
\section{Introduction}

Recommender system is an indispensable technology in this big data era \cite{lu2015recommender}. It help us to find personalized products or services from the ever-increasing information overloads and make our life more efficient and focused. Collaborative Filtering (CF) and matrix based approach are a dominant recommendation approach For the past decades. Most of the works require user-item interactions to be presented for training. When it comes to real-world application, a warming up period is commonly required for CF based approach to over come the cold start and data sparsity problems, due to continuously evolving data.

Dimensionality reduction method, by projecting user/item features into a latent space, is commonly used in model based CF approach. However, a routine end to end re-train process is commonly required, and can be inefficient and costly. Such methods are not suited in dealing brand new items, which have a profound use case in real-world scenarios. To deal with this problem, many fusion approach is developed via content or social network based approach \cite{mao2016multirelational,wang2016member}. Furthermore, mapping user-item interactions into graph data structure enables researchers to further adding user/item feature related information with the Heterogeneous Information Network (HIN). Consequently, item/users' representation can be learned based on the graph local information as well as its global structure. For example, Node2vec \cite{grover2016node2vec} is an effective feature learning approach using network sampling strategy. 

Though there are a number of research had shown fusion approach can help with sparsity problem on rating predictions, the implicit recommendation problem is seldom explored. Optimizing CTR (Click Through Rate) without leaving ratings has a much wider real-world applications. Random walks based presentation learning are known to be more biased to highly connected nodes. \cite{sun2011pathsim} during the un-supervised learning process. User can have little control of how nodes concurrency is being associated. 

In this paper we propose a Meta-path Embedding based Recommendation (MERec), a novel approach that utilizing meta-path based presentation learning on HIN. This approach is primarily focusing on solving the cold start problems when data is continuously evolving. The model is also designed to be adaptive to newly emerged data, so that recommendation could be made even there is zero user-items interaction available. By translating domain experts relevancy rules into meta-path set, our approach allow a controlled random walk to reduce computation intensity and improve on end result. 

We Identify the paper contributions as following:
\begin{itemize}
    \item a novel fusion approach to effectively handle fresh data problem in implicit recommendation problem;
    \item a controlled approach that capable in incorporating expert knowledge into training process;
    \item a framework that can effectively reduce end to end retraining interval when data keep evolving overtime;
\end{itemize}

The rest of this paper is constructed as follows: In Section II, we discuss the related research and relevant definition that helped our research. In Section III, we explain our framework and related algorithms. In Section IV, we show our experiment design and result comparison with other models in different cold and non-cold start scenario. Lastly In Section V, we discuss our finding and and explore future improve directions.