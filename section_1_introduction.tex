%!TEX root = ./main.tex
\section{Introduction}

Recommender system is an indispensable technology in this big data era \cite{lu2015recommender}. It helps us to find personalized products or services from the ever-increasing information and make our life more efficient and focused. Collaborative Filtering (CF) including the famous matrix factorization methods are a dominant recommendation approach for the past decades. These methods predict user's preference according to user's history records. Consequently, a warming up period is commonly required for CF based approach to over come the data sparsity problems, due to continuously evolving data.

Traditionally, matrix factorization approach projects user/item features into a latent space, then calculate user-item similarity based on their condensed latent vector values. However, such methods are not suited in dealing brand new items, which have a profound use case in real-world scenarios. A known workaround is running a routine end to end re-train process to pick up new interactions, which can be inefficient and costly. Content-based or hybrid recommendation approaches are one way to overcome the problem. It is proved to be challenging when dealing large categorical features, such as exponential computation complexity increase, as well as introducing noise.

Recently, many embedding methods are developed via Heterogeneous Information Network (HIN) based approaches \cite{mao2016multirelational,wang2016member}. Graph data structure enables continuously adding user/item feature related information. The embedding process is to learn item/users' representations based on nodes' local and global structure, via propagation process such as random walk, such as Node2vec \cite{grover2016node2vec} and DeepWalk \cite{perozzi2014deepwalk}. However, random walks based presentation learning are known to be more biased to highly connected nodes \cite{sun2011pathsim}. This problem is alleviated by meta-Path based approaches \cite{dong2017metapath2vec} especially for recommendation with explicit feedbacks. However, these methods emphasize on the data sparsity problem and they cannot deal with brand new items. Moreover, many real-world applications are focusing on implicit feedback, such as optimizing click through rate. This remains to be a challenging problem to be solved 

In this paper we propose a Meta-path Embedding based Recommendation (MERec), a novel approach that utilizing meta-path based presentation learning on HIN. This approach is primarily focusing on solving the cold start problems when data is continuously evolving. It is also designed to be adaptive to newly emerged data, so that recommendation could be made even there is zero user-items interaction available. By translating domain experts relevancy rules into meta-path set, our approach allow a controlled random walk to reduce computation intensity and improve on end result. 

We Identify the paper contributions as following:
\begin{itemize}
    \item A novel meta-path based embedding method to effectively handle cold-start problem in recommendation with implicit feedback;
    \item A controlled approach that capable of incorporating expert knowledge into training process;
    \item Experiments are conducted on ....;
\end{itemize}

The rest of this paper is constructed as follows. In Section 2, we discuss the related research and relevant definition that helped our research. In Section 3, we explain our framework and related algorithms. In Section 4, we show our experiment design and result compared with other baselines in cold-start scenario and sufficient data scenario. Lastly, we give conclusion and future directions in Section 5.